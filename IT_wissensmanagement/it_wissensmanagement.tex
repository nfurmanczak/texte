\documentclass{article}
\usepackage[margin=3cm]{geometry}
\usepackage{natbib}
\usepackage{url}
\bibliographystyle{plainnat}

\title{Wissensmanagement IT}
\author{Nikolai Furmanczak}
\date{} 

\begin{document}
\section*{Wissensmanagement in der IT}
Die IT, unabhängig welche Teildisziplin, gehört wohl unumstritten zu den Branchen die am schnelllebigsten sind. Dies hat sich insbesondere in den Themen Cloud-Computing oder Künstliche Inteligenz (KI) in den letzten Jahren deutlich wiedergespiegelt. Neben der schnelligkeit tendiert die IT aber auch zu einer besonderen komplexität. Infrastukturen bestehen immer aus immer mehr Komponenten und in einem Softwareprojekt gibt es immer mehr Abhängigkeiten zu anderen Frameworks oder Bibiliotheken. Vor allem JavaScript Entwickler werden den letzten Punkt kennen. 
\par
Diese beiden Punkte - komplexität und schnelligkeit - führen zu einer immer größer werdenen Flut an Informationen und Wissen die Personen in diesem Umfeld bewältigen müssen. Damit wichtiges Wissen nicht verloren geht und auch innerhalb einer Firma auch geteilt wird, sollten sich Unternehmen aber auch Manager mit dem Thema Wissensmanagement beschäftigen. Um direkt eins vorweg zu nehmen: Das alleinige pflegen eines Wikis zählt nicht als Wissensmanagement.\par
\bigskip 
Bevor wir uns mit dem Thema Wissensmanagement befassen müssen wir den Begriff "Wissen" einmal definieren. Leider gibt es in der Literatur keine einheitliche Definition für Wissen \citep{ReinmannRothmeierMandl2000}

\bibliography{it_wissensmanagement}
\end{document}